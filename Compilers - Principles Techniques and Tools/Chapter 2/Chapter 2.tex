\documentclass{article}

    % Input language encoding
    %\usepackage[utf8]{inputenc}
   
    % Output languages
    %\usepackage[greek, english]{babel}
    % \usepackage{alphabeta}
    
    % Fonts
    %\usepackage[T1,LGR]{fontenc}
    \usepackage{lmodern}

    % Images
    \usepackage{graphicx}
    \usepackage{float}
    \usepackage{caption}
    \usepackage{subcaption}

    % Math
    \usepackage{amsmath}

    % Paragraph Formatting
    \usepackage{parskip}

    % Code
    \usepackage{listings}
    \usepackage{fancyvrb}

    % Different Enumerations
    \usepackage{enumitem}

    % Trees
    \usepackage{qtree}

    % Links
    \usepackage{hyperref}

    % Setup
    \hypersetup{
        colorlinks=true,
        linkcolor=blue,
        filecolor=magenta,      
        urlcolor=cyan,
    }

    \urlstyle{same}
        
    \DeclareMathSizes{10}{10}{10}{10}
    \setlength{\parindent}{0cm}

    \title{Chapter 2}

\begin{document}

\pagenumbering{gobble}
\date{}
\author{}

\maketitle

\section*{2.2.7 Excercises for Section 2.2}

\subsection*{Excercise 2.2.1}
\begin{enumerate}[label=\alph*)]
    \item
    \begin{gather*}
        \begin{align*}
            &\textit{S} \rightarrow \textit{S1}\;\textit{S2}\;* \\
            &\textit{S1} \rightarrow \textit{S3}\;\textit{S4}\;+ \\
            &\textit{S2} \rightarrow a \\
            &\textit{S3} \rightarrow a \\
            &\textit{S4} \rightarrow a \\
        \end{align*}
    \end{gather*}
    \item 
    \Tree [.S [.S [.S a ][.S a ] + ][.S a ] * ]
    \item
    The language generated is the post-fix notation of numbers with multiplication and addition operands.
\end{enumerate}

\subsection*{Excercise 2.2.2}
\begin{enumerate}[label=\alph*)]
    \item 
    The language created is $0^{n}1^{n}$, where $n\in N^{*}$.
    \item 
    This language is the prefix notation of the addition and difference of the digit a.
    \item
    The language is $[(^{n})^{n}]^{m}$, where $m,n\in N$ and for every diffferent m the n is different, so closed parenthesis of any depth and length.
    \item The language is $(a^{n}b^{n})^{m}$, where $m,n\in N$ and for different m, the n is also different. So different sequences of a and b where both letter have the same number of appearances.
    \item
    This is a grammar to create regular languages \href{https://en.wikipedia.org/wiki/Regular_language#Formal_definition}{(Wikipedia link)}.
\end{enumerate}

\subsection*{2.2.3}
The grammars that are ambiguous are:

\begin{itemize}
\item Grammar c: Creating the string "()()" can be done in two ways

Way A:

\Tree [.S [[.S [.S e ] ( [.S e ] ) [.S e ] ] ( [.S e ] ) [.S e ]]]

Way B:

\Tree [.S [.S e ] ( [.S e ] ) [.S [.S e ] ( [.S e ] ) [.S e ]]]

\item Grammar d: Creating the string "abab" can be done in two ways:

Way A:

\Tree [.S a [.S [.S e ] b [.S e ] a ] b [.S e ]]

Way B:

\Tree [.S a [.S e ] b [.S  a [.S e ] b [.S e ]]]

\item Grammar e: Creating the string "a a+a" can be done in two ways:

Way A:

\Tree [.S [.S [.S a ] [.S a ]] + [.S a ]]

Way B:

\Tree [.S [.S a ] [.S [.S a ] + [.S a ]]] 
\end{itemize}

\subsection*{Excercise 2.2.4}

\begin{enumerate}[label=\alph*)]
    \item This is called reverse polish notation \href{https://en.wikipedia.org/wiki/Reverse_Polish_notation}{(Wikipedia Link)}
    \begin{gather*}
        \textit{expr} \rightarrow \textit{expr}\;\textit{expr}\;\text{op} \mid \text{digit}
    \end{gather*}
    \item 
\end{enumerate}
\end{document}