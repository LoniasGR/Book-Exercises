\documentclass{article}

    % Input language encoding
    %\usepackage[utf8]{inputenc}
   
    % Output languages
    %\usepackage[greek, english]{babel}
    % \usepackage{alphabeta}
    
    % Fonts
    \usepackage[T1]{fontenc}
    \usepackage{lmodern}

    % Images
    \usepackage{graphicx}
    \usepackage{float}
    \usepackage{caption}
    \usepackage{subcaption}

    % Math
    \usepackage{amsmath}

    % Paragraph Formatting
    \usepackage{parskip}

    % Code
    \usepackage{listings}

    % Arrays
    \renewcommand{\arraystretch}{2}


        

    \DeclareMathSizes{10}{10}{10}{10}
    \setlength{\parindent}{0cm}

    \title{Problem 1-1}

\begin{document}

\pagenumbering{gobble}
\date{}
\author{}

\maketitle

For each algorithm we know that it takes f(n) = y microseconts (i.e for 10 inputs, lg 10 takes 1 microsecond).
So f(n) = y.
To figure out how many inputs the algorithms can solve, we do the following:

We inverse the function and then replace y with 
$10^{6}, 6 * 10^{7}, 36 * 10^{8}, 864 * 10^{8}, 2592 * 10^{9}, 31536 * 10^{10}, 31536 * 10^{12}$ respectively.

So we have the following inverses:

\begin{itemize}
    \item For $\lg{n} = y$, we get $10^y = n$.
    \item For $\sqrt{n} = y$, we get $y^{2} = n$.
    \item For $n = y$, we get $y = n$.
    \item For $n*\lg{n} = y$, we cannot get inverse. We will use Newton's method for this (Numerical Analysis), in a Python script.
    \item For $n^{2} = y$, we get $\sqrt{y} = n$.
    \item For $n^{3} = y$, we get $\sqrt[3]{y} = n$.
    \item For $2^{n} = y$, we get $\log{y} = n$.
    \item For $n! = y$, we will have to divide by consecutive numbers, starting from 1 to find $n$.
    
\end{itemize}


\textit{Convention:} A month has 30 days and a year 365.

The Python code we used is the following (can also be found in newton\_method.py):

\begin{lstlisting}[language=Python]
    import math

    N = int(input("Equals to: "))
    
    def create_estimate (x):
        "We estimate as x/logx to begin the search for a root."
        return x/math.log(x,10)
    
    def f (x):
        "x*lgx function"
        return x * math.log(x, 10)
    
    def f_der (x):
        "lgx + 1/ln10"
        return math.log(x,10) + 1/math.log(10)
    
    estimate = create_estimate(N)
    print ("Estimate is ", estimate)
    new_estimate = estimate - ((f(estimate) - N)/f_der(estimate))
    
    
    while (abs(new_estimate - estimate) > 0.1 ):
        estimate = new_estimate
        new_estimate = estimate = estimate - ((f(estimate) - N)/f_der(estimate))
    
    print (new_estimate)
    
\end{lstlisting}

This is the final result: 

\hskip-2.0cm
\begin{tabular}{ |c|c|c|c|c|c|c|c| }
    \hline
        & 1 sec. & 1 min. & 1 hr. & 1 day. & 1 mon. & 1 yr. & 1 cent. \\
    \hline
    $\lg{n}$ & $10^{10^{6}}$ & $10^{6*10^{7}}$ & $10^{36*10^{8}}$ & $10^{864*10^{8}}$ & $10^{2592*10^{9}}$ & $10^{31536*10^{10}}$ & $10^{31536 * 10^{12}}$\\
    \hline
    $\sqrt{n}$ & $10^{12}$ & $36 * 10^{14}$ & $1296 * 10^{16}$ & $746496 * 10^{16}$ & $6718464 * 10^{18}$ & $994519296 * 10^{20}$ & $994519296 * 10^{24}$\\ 
    \hline
    $n$ & $10^{6}$ & $6*10^{7}$ & $36*10^{8}$ & $864 * 10^{8}$ & $2592 * 10^{9}$ & $31536*10^{10}$ & $31536*10^{12}$\\
    \hline
    $n*\lg{n}$ & $189481$ & $8649296$ & $417596733$ & $8692884131$ & $228202745102$ & $23582565190405$ & $2059329031596633$\\
    \hline
    $n^{2}$ & $1000 $ & $7746$ & $60000$ & $293940$ & $161 * 10^{3}$ & $17758 * 10^{3}$ & $17758 * 10^{4}$\\
    \hline
    $n^{3}$ & $100$ & $391$ & $1532$ & $4420$ & $13737$ & $68067$ & $315940$ \\
    \hline
    $2^{n}$ & $19$ & $25$ & $31$ & $36$ & $41$ & $48$ & $54$\\
    \hline
    $n!$ & $9$ & $11$ & $12$ & $13$ & $15$ & $16$ & $18$\\
    \hline

\end{tabular}


\end{document}