\documentclass{article}

    % Input language encoding
    %\usepackage[utf8]{inputenc}
   
    % Output languages
    %\usepackage[greek, english]{babel}
    % \usepackage{alphabeta}
    
    % Fonts
    %\usepackage[T1,LGR]{fontenc}
    \usepackage{lmodern}

    % Images
    \usepackage{graphicx}
    \usepackage{float}
    \usepackage{caption}
    \usepackage{subcaption}

    % Math
    \usepackage{amsmath}

    % Paragraph Formatting
    \usepackage{parskip}

    % Code
    \usepackage{listings}

        

    \DeclareMathSizes{10}{10}{10}{10}
    \setlength{\parindent}{0cm}

    \title{Paragraph A.1 Excercises}

\begin{document}

\pagenumbering{gobble}
\date{}
\author{}

\maketitle

\section*{A.1-1}
From linearity property of the summations, we know that,\begin{gather*}
    \sum_{k=1}^{n}{(ca_{k} + b_{k})} = c\sum_{k=1}^{n}{a_{k}} + \sum_{k=1}^{n}{b_{k}}
\end{gather*} 

So considering $a_{k} = k$, $c = 2$ and $b_{k} = 1$ we have:
\begin{gather}
    \sum_{k=1}^{n}{(2k + 1)} = 2\sum_{k=1}^{n}{k} + \sum_{k=1}^{n}{1} \label{eq1}
\end{gather} 

We know from theory that $\sum_{k=1}^{n}{k} = \frac{1}{2}n(n+1)$ and that $\sum_{k=1}^{n}{1} = n$.

So in total we have:
\begin{gather*}
    (\ref{eq1}) = 2(\frac{1}{2}n(n+1)) + n = n^{2} + 2n
\end{gather*}

So the final result is $n^{2} + 2n$.

\section*{A.1-2}
We know that the harmonic series is 
\begin{gather*}
    H_{n} = \sum_{k=1}^{n}{\frac{1}{k}} = \ln{n} + O(1)
\end{gather*}
We want to calculate 
\begin{gather*}
    \sum_{k=1}^{n}{\frac{1}{2k-1}}, 
\end{gather*}    
which means that we calculate for all odd k from 1 to 2n, which can be written as 
\begin{gather*}
    \sum_{\text{k is odd}}^{2n}{\frac{1}{k}}
\end{gather*}
It is obvious that:
\begin{gather}
    \sum_{1}^{2n}{\frac{1}{k}} = \sum_{\text{k is odd}}^{2n}{\frac{1}{k}} + \sum_{\text{k is even}}^{2n}{\frac{1}{k}} \label{sum2}
\end{gather}

We can see that:
\begin{gather}
    \sum_{\text{k is even}}^{2n}{\frac{1}{k}} = \sum_{1}^{n}{\frac{1}{2k}} = \frac{1}{2} \sum_{1}^{n}{\frac{1}{k}} = \frac{1}{2}(\ln{n} + O(1)) \label{sum3} 
\end{gather}

For $\sum_{1}^{2n}{\frac{1}{k}}$ we also have:
\begin{gather}
    \sum_{1}^{2n}{\frac{1}{k}} = \ln{2n} + O(1) = \ln{2} + \ln{n} + O(1) = \ln{n} + O(1) \label{sum4}
\end{gather}

From \ref{sum2}, \ref{sum3} and \ref{sum4} we get:
\begin{gather*}
    \sum_{\text{k is odd}}^{2n}{\frac{1}{k}} = \sum_{1}^{2n}{\frac{1}{k}} - \sum_{\text{k is even}}^{2n}{\frac{1}{k}} = \ln{n} - \frac{1}{2}\ln{n} + O(1) = \frac{1}{2}\ln{n} + O(1) =\\
    \ln{\sqrt{n}} + O(1)
\end{gather*}

\section*{A.1-3}
We know that the geometric series is $\sum_{k=0}^{\infty}{x^{k}} = \frac{1}{1 - x}$, for $0 < |x| < 1$.

We can differentiate both parts of this series, so we get:
\begin{gather}
    \sum_{k=0}^{\infty}{kx^{k-1}} = \frac{1}{(1 - x)^2},0< |x| < 1 \label{eq2}
\end{gather}
Differantiating again, we have:
\begin{gather*}
    \begin{align*}
        \sum_{k=0}^{\infty}{k(k-1)x^{k-2}} = \frac{-2(-1)}{(1 - x)^3} \Rightarrow \\
        \sum_{k=0}^{\infty}{(k^{2}-k)x^{k-2}} = \frac{2}{(1-x)^{3}}
    \end{align*}
\end{gather*}

By multiplying with $x^{2}$ and using the linearity property, we have:

\begin{gather}
    \sum_{k=0}^{\infty}{(k^{2}-k)x^{k}} = \sum_{k=0}^{\infty}{k^{2}x^{k}} - \sum_{k=0}^{\infty}{kx^{k}} = \frac{2x^{2}}{(1-x)^{3}} \label{eq3}
\end{gather}

By multiplying equation (\ref{eq2}) with x, we get:
\begin{gather}
    \sum_{k=0}^{\infty}{kx^{k}} = \frac{x}{(1 - x)^2} \label{eq4}
\end{gather}

So from equations (\ref{eq3}) and (\ref{eq4}), we get:
\begin{gather*}
    \sum_{k=0}^{\infty}{k^{2}x^{k}} - \frac{x}{(1 - x)^2} = \frac{2x^{2}}{(1-x)^{3}} \Rightarrow \\
    \sum_{k=0}^{\infty}{k^{2}x^{k}} = \frac{2x^{2}}{(1-x)^{3}} + \frac{x}{(1 - x)^2} = \frac{2x^{2} + x(1-x)}{(1-x)^{3}} = \\
    \frac{x^{2} + x}{(1-x)^{3}} \Rightarrow \\
    \sum_{k=0}^{\infty}{k^{2}x^{k}} = \frac{x(1+x)}{(1-x)^{3}}, 0 < x| < 1
\end{gather*}


\end{document}