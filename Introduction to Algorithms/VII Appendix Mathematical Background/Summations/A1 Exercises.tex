\documentclass{article}

    % Input language encoding
    %\usepackage[utf8]{inputenc}
   
    % Output languages
    %\usepackage[greek, english]{babel}
    % \usepackage{alphabeta}
    
    % Fonts
    %\usepackage[T1,LGR]{fontenc}
    \usepackage{lmodern}

    % Images
    \usepackage{graphicx}
    \usepackage{float}
    \usepackage{caption}
    \usepackage{subcaption}

    % Math
    \usepackage{amsmath}

    % Paragraph Formatting
    \usepackage{parskip}

    % Code
    \usepackage{listings}

        

    \DeclareMathSizes{10}{10}{10}{10}
    \setlength{\parindent}{0cm}

    \title{Paragraph A.1 Exercises}

\begin{document}

\pagenumbering{gobble}
\date{}
\author{}

\maketitle

\section*{A.1-1}
From linearity property of the summations, we know that,\begin{gather*}
    \sum_{k=1}^{n}{(c_{1}a_{k} + c_{2}b_{k})} = c_{1}\sum_{k=1}^{n}{a_{k}} + c_{2}\sum_{k=1}^{n}{b_{k}}
\end{gather*} 

So considering $a_{k} = k$, $c_{1} = 2$, $c_{2} = -1$ and $b_{k} = 1$ we have:
\begin{gather}
    \sum_{k=1}^{n}{(2k + 1)} = 2\sum_{k=1}^{n}{k} - \sum_{k=1}^{n}{1} \label{eq1}
\end{gather} 

We know from theory that $\sum_{k=1}^{n}{k} = \frac{1}{2}n(n+1)$ and that $\sum_{k=1}^{n}{1} = n$.

So in total we have:
\begin{gather*}
    (\ref{eq1}) = 2(\frac{1}{2}n(n+1)) - n = n^{2}
\end{gather*}

So the final result is $n^{2}$.

\section*{A.1-2}
We know that the harmonic series is 
\begin{gather*}
    H_{n} = \sum_{k=1}^{n}{\frac{1}{k}} = \ln{n} + O(1)
\end{gather*}
We want to calculate 
\begin{gather*}
    \sum_{k=1}^{n}{\frac{1}{2k-1}}, 
\end{gather*}    
which means that we calculate for all odd k from 1 to 2n, which can be written as 
\begin{gather*}
    \sum_{\text{k is odd}}^{2n}{\frac{1}{k}}
\end{gather*}
It is obvious that:
\begin{gather}
    \sum_{1}^{2n}{\frac{1}{k}} = \sum_{\text{k is odd}}^{2n}{\frac{1}{k}} + \sum_{\text{k is even}}^{2n}{\frac{1}{k}} \label{sum2}
\end{gather}

We can see that:
\begin{gather}
    \sum_{\text{k is even}}^{2n}{\frac{1}{k}} = \sum_{1}^{n}{\frac{1}{2k}} = \frac{1}{2} \sum_{1}^{n}{\frac{1}{k}} = \frac{1}{2}(\ln{n} + O(1)) \label{sum3} 
\end{gather}

For $\sum_{1}^{2n}{\frac{1}{k}}$ we also have:
\begin{gather}
    \sum_{1}^{2n}{\frac{1}{k}} = \ln{2n} + O(1) = \ln{2} + \ln{n} + O(1) = \ln{n} + O(1) \label{sum4}
\end{gather}

From \ref{sum2}, \ref{sum3} and \ref{sum4} we get:
\begin{gather*}
    \sum_{\text{k is odd}}^{2n}{\frac{1}{k}} = \sum_{1}^{2n}{\frac{1}{k}} - \sum_{\text{k is even}}^{2n}{\frac{1}{k}} = \ln{n} - \frac{1}{2}\ln{n} + O(1) = \frac{1}{2}\ln{n} + O(1) =\\
    \ln{\sqrt{n}} + O(1)
\end{gather*}

\section*{A.1-3}
We know that the geometric series is $\sum_{k=0}^{\infty}{x^{k}} = \frac{1}{1 - x}$, for $0 < |x| < 1$.

We can differentiate both parts of this series, so we get:
\begin{gather}
    \sum_{k=0}^{\infty}{kx^{k-1}} = \frac{1}{(1 - x)^2},0< |x| < 1 \label{eq2}
\end{gather}
Differantiating again, we have:
\begin{gather*}
    \begin{align*}
        \sum_{k=0}^{\infty}{k(k-1)x^{k-2}} = \frac{-2(-1)}{(1 - x)^3} \Rightarrow \\
        \sum_{k=0}^{\infty}{(k^{2}-k)x^{k-2}} = \frac{2}{(1-x)^{3}}
    \end{align*}
\end{gather*}

By multiplying with $x^{2}$ and using the linearity property, we have:

\begin{gather}
    \sum_{k=0}^{\infty}{(k^{2}-k)x^{k}} = \sum_{k=0}^{\infty}{k^{2}x^{k}} - \sum_{k=0}^{\infty}{kx^{k}} = \frac{2x^{2}}{(1-x)^{3}} \label{eq3}
\end{gather}

By multiplying equation (\ref{eq2}) with x, we get:
\begin{gather}
    \sum_{k=0}^{\infty}{kx^{k}} = \frac{x}{(1 - x)^2} \label{eq4}
\end{gather}

So from equations (\ref{eq3}) and (\ref{eq4}), we get:
\begin{gather*}
    \sum_{k=0}^{\infty}{k^{2}x^{k}} - \frac{x}{(1 - x)^2} = \frac{2x^{2}}{(1-x)^{3}} \Rightarrow \\
    \sum_{k=0}^{\infty}{k^{2}x^{k}} = \frac{2x^{2}}{(1-x)^{3}} + \frac{x}{(1 - x)^2} = \frac{2x^{2} + x(1-x)}{(1-x)^{3}} = \\
    \frac{x^{2} + x}{(1-x)^{3}} \Rightarrow \\
    \sum_{k=0}^{\infty}{k^{2}x^{k}} = \frac{x(1+x)}{(1-x)^{3}}, 0 < |x| < 1
\end{gather*}

\section*{A.1-4}
We want to show that 
\begin{equation*}
    \sum_{k=0}^{\infty}{\frac{(k-1)}{2^{k}}} = 0
\end{equation*}.

From linearity property we know that 
\begin{equation} \label{eq:5}
    \sum_{k=0}^{\infty}{\frac{(k-1)}{2^{k}}} =    \sum_{k=0}^{\infty}{\frac{k}{2^{k}}} - \sum_{k=0}^{\infty}{\frac{1}{2^{k}}} 
\end{equation}

We know that the geometric series is $\sum_{k=0}^{\infty}{x^{k}} = \frac{1}{1 - x}$, for $0 < |x| < 1$.

By replacing x with $\frac{1}{2}$, we get 
\begin{gather*}
    \begin{align*}
        &\sum_{k=0}^{\infty}{(\frac{1}{2})^{k}} = \frac{1}{1 - \frac{1}{2}} \Rightarrow\\
        &\sum_{k=0}^{\infty}{(\frac{1}{2^{k}})} = 2
    \end{align*}
\end{gather*}

That way we have found the value of the second sum of Equation \ref{eq:5}.

Moreover, we know that by differentiating the geometric series and multiplying by x, we get $\sum_{k=0}^{\infty}{kx^{k}} = \frac{x}{(1 - x)^2}$. Again, by replacing x with $\frac{1}{2}$, we get:
\begin{gather*}
    \begin{align*}
        &\sum_{k=0}^{\infty}{k(\frac{1}{2})^{k}} = \frac{\frac{1}{2}}{(1 - \frac{1}{2})^2} \Rightarrow \\
        &\sum_{k=0}^{\infty}{\frac{k}{2^{k}}} = \frac{\frac{1}{2}}{(\frac{1}{4})} = 2
    \end{align*}
\end{gather*}

So no we have found the value of the first sum of Equation \ref{eq:5}.

Now it is easy to see that the full sum of Equation \ref{eq:5} equals with $2 - 2 = 0$.

\section*{A.1-5}
We need to evaluate the sum $\sum_{k=1}^{\infty}{(2k+1)x^{2k}}$.

Because of the linearity property, we have:
\begin{equation} \label{eq:6}
    \sum_{k=1}^{\infty}{(2k+1)x^{2k}} = 2 * \sum_{k=1}^{\infty}{k x^{2k}} - \sum_{k=1}^{\infty}{x^{2k}} 
\end{equation}

We know that the geometric series is $\sum_{k=0}^{\infty}{y^{k}} = \frac{1}{1 - y}$, for $0 < |y| < 1$.

By replacing y with $x^{2}$, we have:

\begin{align*}
    &\sum_{k=0}^{\infty}{(x^{2})^{k}} = \frac{1}{1-x^{2}} \Rightarrow\\
    &\sum_{k=0}^{\infty}{x^{2k}} = \frac{1}{1-x^{2}} 
\end{align*}

We want our sum to start from 1, and the value of the sum for $k = 0$ is 1:
\begin{align*}
    &\sum_{k=0}^{\infty}{x^{2k}} = \frac{1}{1-x^{2}} \Rightarrow \\
    &\sum_{k=1}^{\infty}{x^{2k}} + 1 = \frac{1}{1-x^{2}} \Rightarrow \\
    &\sum_{k=1}^{\infty}{x^{2k}} = \frac{1}{1-x^{2}} - \frac{1-x^{2}}{1-x^{2}} \Rightarrow \\
    &\sum_{k=1}^{\infty}{x^{2k}} = \frac{x^{2}}{1-x^{2}}
\end{align*}
So, we have found the value of the second sum of Equation \ref{eq:6}.

Similarly we know that if we differentiate the geometric series and multiply by y, we get $\sum_{k=0}^{\infty}{ky^{k}} = \frac{y}{(1 - y)^2}$.

By replacing y with $x^{2}$, we have:
\begin{align*}
    \sum_{k=0}^{\infty}{k(x^{2})^{k}} = \frac{x^{2}}{(1-x^{2})^{2}} \Rightarrow\\
    \sum_{k=0}^{\infty}{kx^{2k}} = \frac{x^{2}}{(1-x^{2})^{2}} 
\end{align*}

Again, we want the sum to start from 1, but here the value of the sum for $k = 0$ is 0, so we have:
\begin{equation*}
    \sum_{k=0}^{\infty}{kx^{2k}} = \sum_{k=1}^{\infty}{kx^{2k}} = \frac{x^{2}}{(1-x^{2})^{2}} 
\end{equation*}


So, we have found the value of the first sum of Equation \ref{eq:6}.

In total we have:
\begin{equation*}
    \sum_{k=1}^{\infty}{(2k+1)x^{2k}} = \frac{2x^{2}}{(1-x^{2})^{2}} + \frac{x^{2}}{1-x^{2}} =  \frac{2x^{2}}{(1-x^{2})^{2}}  + \frac{x^{2} - x^{4}}{(1 - x^{2})^{2}} = \frac{3x^{2} - x^{4}}{(1-x^{2})^{2}} 
\end{equation*}

\section*{A.1-6}
We want to prove that: $\sum_{k=1}^{n}{O(g_{k}(i))} = O(\sum_{k=1}^{n}{g_{k}(i)})$ by using the linearity property.

Let's assume that there are functions $g_{k}(i)$, so that for each i, the following holds true $|g_{k}(i)| \leq M_{k}f_{k}(i)$ for a postiive real number $M_{k}$ and for all $i \geq i_{0}$. The we say that $g_{k}(i) = O(f_{k}(i))$. 

We select $c = \underset{1 \leq k \leq n}{max}(M_k)$.

The inequality $|g_{k}(i)| \leq cf_{k}(i)$ still fits under the definition of big-O, since we picked the largest of the $M_{k}$.

So if we sum all the $g_{k}$ from 0 until n, we have,

\begin{align}
    &\sum_{k=1}^{n}{g_{k}(i)} \leq \sum_{k=1}^{n}{M_{k}f_{k}(i)} \Rightarrow\\
    &\sum_{k=1}^{n}{g_{k}(i)} \leq \sum_{k=1}^{n}{cf_{k}(i)} \Rightarrow\\
    &\sum_{k=1}^{n}{g_{k}(i)} = \sum_{k=1}^{n}{O(f_{k}(i))}\label{eq:10}
\end{align}

Because of linearity property, we get:
\begin{equation}\label{eq:11}
    \sum_{k=1}^{n}{cf_{k}(i)} = c\sum_{k=1}^{n}{f_{k}(i)}
\end{equation}

So from Equation \ref{eq:10} and Equation \ref{eq:11}, we have:
\begin{align}
    &\sum_{k=1}^{n}{g_{k}(i)} \leq \sum_{k=1}^{n}{cf_{k}(i)} \Rightarrow \\
    &\sum_{k=1}^{n}{g_{k}(i)} \leq c\sum_{k=1}^{n}{f_{k}(i)} \Rightarrow \\
    &\sum_{k=1}^{n}{g_{k}(i)} = O(\sum_{k=1}^{n}{f_{k}(i)}) \label{eq:12}
\end{align}

So now we see that Equation \ref{eq:10} and Equation \ref{eq:12} are equal, so we proved our original equality, that $\sum_{k=1}^{n}{O(f_{k}(i))} = O(\sum_{k=1}^{n}{f_{k}(i)})$, for any kind of function f.

\section*{A.1-7}

We will use the fact that $\lg{(\prod_{k=1}^{n}{a_k})} = \sum_{k=1}^{n}{\lg{a_k}}$.

We have :

\begin{align*}
    &\lg{(\prod_{k=1}^{n}{2 * 4^k})} = \sum_{k=1}^{n}{\lg{(2 * 4^k)}} = \\
    &\sum_{k=1}^{n}{(\lg 2 + \lg 4^k)} = \\
    &\sum_{k=1}^{n}{(\lg 2 + k\lg 4)} = \sum_{k=1}^{n}{(1 + 2k)} = \\
    &\sum_{k=1}^{n}{1} + 2 \sum_{k=1}^{n}{k} = n + 2 \frac{1}{2}n(n+1) = \\
    &n^2 + 2n 
\end{align*}

By using the fact that $2^{\lg{(\prod_{k=1}^{n}{a_k})}} = \prod_{k=1}^{n}{a_k}$, we have:

\begin{align*}
    &2^{\lg{(\prod_{k=1}^{n}{2 * 4^k})}} = 2^({n^2 + 2n}) = \\
    &2^{n^2} + 4^n
\end{align*}

\section*{A.1-8}

We will expand the product:

\begin{align*}
    &\prod_{k=2}^{n}{(1 - \frac{1}{k^2})} = \prod_{k=2}^{n}{(\frac{k^2 - 1}{k^2})} =\\
    &\prod_{k=2}^{n}{(\frac{(k - 1)(k + 1)}{k^2})} =\\
    &\frac{1 * 3}{2 * 2} * \frac{2 * 4}{3 * 3} * \frac{3 * 5}{4 * 4} . . . \frac{(n-1)(n+1)}{n * n} =\\
    & \frac{1 * (n+1)}{2 * n} = \frac{(n+1)}{2n}
\end{align*}



\end{document}