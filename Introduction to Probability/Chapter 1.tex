\documentclass{article}

    % Input language encoding
    %\usepackage[utf8]{inputenc}
   
    % Output languages
    %\usepackage[greek, english]{babel}
    % \usepackage{alphabeta}
    
    % Fonts
    %\usepackage[T1,LGR]{fontenc}
    \usepackage{lmodern}

    % Images
    \usepackage{graphicx}
    \usepackage{float}
    \usepackage{caption}
    \usepackage{subcaption}

    % Math
    \usepackage{amsmath}

    % Paragraph Formatting
    \usepackage{parskip}

    % Code
    \usepackage{listings}

        

    \DeclareMathSizes{10}{10}{10}{10}
    \setlength{\parindent}{0cm}

    \title{Chapter 1}

\begin{document}

\pagenumbering{gobble}
\date{}
\author{}

\maketitle

\section*{SECTION 1.1 Sets}

\subsection*{Problem 1}
According to the description of the problem, we have:\\
$A = \{2, 4, 6\}$ and $B = \{4, 5, 6\}$\\
$\Omega = \{1, 2, 3, 4, 5, 6\}$

So we get the following for the 1st of DeMorgan's law:
\begin{gather*}
    (A \cup B)^{c} =  (\{2, 4, 6\} \cup \{4, 5, 6 \})^{c} = \{2, 4, 5, 6\}^{c} = \{1, 3\} \\
    A^{c} \cap B^{c} = \{2, 4, 6\}^{c} \cap \{4, 5, 6\}^{c} = \{1, 3, 5\} \cap \{1, 2, 3\} = \{1, 3\} 
\end{gather*}

So we see that $(A \cup B)^{c} = A^{c} \cap B^{c}$ holds true.

For the second DeMorgan's law, we have:
\begin{gather*}
    (A \cap B)^{c} = (\{2, 4, 6\} \cap \{4, 5, 6\})^{c} = \{4, 6\}^{c} = \{1, 2, 3, 5\} \\
    A^{c} \cup B^{c} = \{2, 4, 6\}^{c} \cup \{4, 5, 6\}^{c} = \{1, 3, 5\} \cup \{1, 2, 3\} = \{1, 2, 3, 5\}
\end{gather*}

So again we see that those two are equal.

\subsection*{Problem 2}

\subsubsection*{(a)}

We will start from $A^{c}$ and build the equation:

\begin{gather*}
    A^{c} = \\
    A^{c} \cap \Omega = \\
    A^{c} \cap (B \cup B^{c}) = \\
    (A^{c} \cap B) \cup (A^{c} \cap B^{c})
\end{gather*}

In a similar fashion we do $B^{c}$:


\begin{gather*}
    B^{c} = \\
    B^{c} \cap \Omega = \\
    B^{c} \cap (A \cup A`^{c}) = \\
    (B^{c} \cap A) \cup (B^{c} \cap A^{c}) = \\
    (A \cap B^{c}) \cup (A^{c} \cap B^{c})
\end{gather*}

\subsubsection*{b}

Starting from the results of (a), we get:

\begin{gather*}
    A^{c} \cup B^{c} \stackrel{\text{\tiny{DeMorgan}}}{=} \\
    (A \cap B)^{c} = (A^{c} \cap B) \cup (A^{c} \cap B^{c}) \cup (A \cap B^{c}) \cup (A^{c} \cap B^{c}) = \\
    (A^{c} \cap B) \cup (A^{c} \cap B^{c}) \cup (A^{c} \cap B^{c})
\end{gather*}

\subsubsection*{c}

From the description, we know that \\
$A = \{1, 3, 5\}$ and $B = \{1, 2, 3\}$.

Based on that, we have:

\begin{gather*}
    (A \cap B)^{c} = \\
    (\{1, 3, 5\} \cap \{1, 2, 3\})^{c} = \\
    \{1, 3\}^{c} = \\
    \{2, 4, 5, 6\}
\end{gather*}

Similarly we have:

\begin{gather*}
    (A^{c} \cap B) \cup (A^{c} \cap B^{c}) \cup (A \cap B^{c}) = \\
    (\{1, 3, 5\}^{c} \cap \{1, 2, 3\}) \cup (\{1, 3, 5\}^{c} \cap \{1, 2, 3\}^{c}) \cup (\{1, 3, 5\} \cap \{1, 2, 3\}^{c}) = \\
    (\{2, 4, 6\} \cap \{1, 2, 3\}) \cup (\{2, 4, 6\} \cap \{4, 5, 6\}) \cup (\{1, 3, 5\} \cap \{4, 5, 6\}) = \\
    \{2\} \cup \{4, 6\} \cup \{5\} = \\
    \{2, 4, 5, 6\}
\end{gather*}

So the equation in (b) holds true.

\subsection*{SECTION 1.2 Probabilistic Models}

\subsubsection*{Problem 5}

We call A the fact that they are geniouses, so $P(A) = 0.6$. \\
We call B the fact that they love chocolate, so $P(B) = 0.7$. \\
We also know that $P(A \cap B) = 0.4$.

We want to find out $P(A^{c} \cap B^{c})$.

From DeMorgan's law, we know that $A^{c} \cap B^{c} = (A \cup B)^{c}$.

We also know that $P(A \cup B) = P(A) + P(B) - P(A \cap B)$.

Finally $P((A \cup B)^{c}) = 1 - P(A \cup B)$.

So e get that $P(A^{c} \cap B^{c}) =1 - 0.6 + 0.7 - 0.4 = 1 - 0.9 = 0.1$.


\subsubsection*{Problem 6}

We have a 6-sided dice. Every even side has double the chance of an odd side. So we consider as having $3 + 6$ sides. Then one odd side has a chance of $P(O) = \frac{1}{9}$, while each even side has a chance of $P(E) = \frac{2}{9}$.

So for $P(Roll < 4)$, we have:
\begin{gather*}
    P(Roll < 4) = P(1) + P(2) + P(3) = \\
    \frac{1}{9} + \frac{2}{9} + \frac{1}{9} = \\
    \frac{4}{9}
\end{gather*}

\subsubsection*{Problem 7}

We name O the fact that the roll is Odd.\\
We name E the fact that the roll is Even.

The sample space for this experiment would be 


$\Omega = \{E, OE, OOE, OOOE, ...\}$

So the probability of it would be $(\frac{1}{4})^{n}$, where n is the tries before getting an even number.

\subsubsection*{Problem 8}

There are three opponents, A, B and C. Without the loss of accuracy, we can assume that $P(A) > P(B) > P(C)$, where P denotes the chance of victory.

There are the following ways to play with all opponents:

\begin{itemize}
    \item A, B, C
    \item A, C, B
    \item B, A, C
    \item B, C, A
    \item C, B, A
    \item C, A, B
\end{itemize}

We have to win two games in a row to succeed. So of course, in order to maximize the chances of winning, by logic, we understand that we want to play with the two easiest opponents in a row, because $P(B)*P(C)$ is the bigger number. So, we are left with the following options:

\begin{itemize}
    \item A, B, C
    \item A, C, B
    \item B, C, A
    \item C, B, A
\end{itemize}

What we see is that if we play first with A and then with B, we have $P(A)*P(B)$, which is smaller than $P(B)*P(C)$ and $P(A)*P(C)$. So we are waisting our chances if we go with the two hardest opponents in a row. So it is better to put C in the middle, so all the two in a row battles have the smallest chance. So we are left with the following options:


\begin{itemize}
    \item A, C, B
    \item B, C, A
\end{itemize}


\subsubsection*{Problem 9}

\subsubsection*{a}

The collection of disjoint events $S_{1},...,S_{n}$ is a partition of $\Omega$.
So if we have any event A, this will be overlapping with the already existing partitions $S_{i}$. 

In other words:

\begin{gather*}
    A = A \cap \Omega \Rightarrow \\
    A = A \cap \bigcup_{i=1}^{n}S_{i} \Rightarrow \\
    A = \bigcup_{i=1}^{n}(A \cap S_{i}) \Rightarrow \\
    P(A) = P(\bigcup_{i=1}^{n}(A \cap S_{i})).
\end{gather*}

Because the sets are disjoint, their intersection with A is still disjoint, so we get:
\begin{gather*}
    P(A) = \sum_{i=1}^{n}P(A \cap S_{i})    
\end{gather*}

\subsubsection*{b}

We try to create a distinct particion of $\Omega$. This is $\{B\cap C^{c}, B^{c} \cap C, B \cap C, B^{c} \cap C^{c}\}$.

So from (a), we have that:

\begin{gather*}
    P(A) = P(A \cap B\cap C^{c}) + P(A \cap B^{c} \cap C) + P(A \cap B \cap C) + P(A \cap B^{c} \cap C^{c})
\end{gather*}

But we can easily see that
\begin{gather*}
    P(A \cap B) + P(A \cap C) = P(A \cap B\cap C^{c}) + P(A \cap B^{c} \cap C) + 2 * P(A \cap B \cap C).
\end{gather*}

So in the end we get $P(A) = P(A \cap B) + P(A \cap C) + P(A \cap B^{c} \cap C^{c}) - P(A \cap B \cap C)$.

\subsubsection*{Problem 10}

We know that $P(A \cup B) = P(A) + P(B) - P(A \cap B)$.

We also see that $P(A \cap B^{c}) = P(A \cup B) - P(B)$.

Similarly $P(A^{c} \cap B) = P(A \cup B) - P(A)$.

We can easily see that these two sets are disjoint, since each gives the probability of exactly one of A or B occuring.

So we have,
\begin{gather*}
    P((A^{c} \cap B) \cup (A \cap B^{c})) =\\
    P(A^{c} \cap B) + P(A \cap B^{c}) = \\
    2 * P(A \cup B) - P(A) - P(B) = \\
    P(A) + P(B) - 2 * P(A \cap B).
\end{gather*}

\subsubsection*{Problem 14}

\subsubsection*{a}

$A = \{11, 22, 33, 44, 55, 66\}$.
Also $N(\Omega) = 36$.
So we get that the chance for doubles is $P(A) = \frac{N(A)}{N(\Omega)} = \frac{6}{36} = \frac{1}{6}$.

\subsubsection*{b}

B is the set of outcomes that result in a sum of 4 or less, so:

\begin{gather*}
    B = \{11, 12, 13, 21, 22, 31\} \\
    P(B) = \frac{6}{36} = \frac{1}{6}
\end{gather*}

According to the conditional probability, we have:

\begin{gather*}
    P(A|B) = \frac{P(A \cap B)}{P(B)} = \\
    \frac{\frac{2}{36}}{\frac{6}{36}} = \frac{2}{6} = \frac{1}{3}
\end{gather*}

\subsubsection*{c}

There are 6 possibilities the first roll to be 6 (second roll can be from 1 to 6) and another 5 for the second roll to be 6 (first roll to be from 1 to 5, the possibility to be the first roll 6 was covered in the first case), for a total of 11.

So $P(C) = \frac{11}{36}$.

\subsubsection*{d}

The probability of the two dice to land on different numbers is $P(D) = P(A^{c}) = \frac{30}{36}$.

We want to find the conditional probability $P(C | D)$.

So,

\begin{gather*}
    P(C|D) = \frac{P(C \cap D)}{P(D)} = \\
    \frac{\frac{10}{36}}{\frac{30}{36}} = \frac{10}{30} = \frac{1}{3}
\end{gather*}

\subsubsection*{Problem 15}



A is the first toss is heads. \\
B is the second toss is heads.

We must compare $P(A \cap B | A)$ with $P((A \cap B)|( A \cup B))$.
 We have:
 \begin{gather*}
    P(A \cap B | A) = \frac{P(A \cap B \cap A)}{P(A)} = \frac{P(A \cap B)}{P(A)} \\
    P((A \cap B)| (A \cup B)) = \frac{P((A \cap B) \cap (A \cup B))}{P(A \cup B)} = \frac{P(A \cap B)}{P(A \cup B)}
 \end{gather*}
We have that $P(A \cup B) \geq P(A)$, which means that the first conditional probability is at least as large.

It seems that it does not matter if the coin is fair or not. 

If the coin is fair:

A is the first toss is heads. \\
B is there is at least one heads toss. \\
C is we have two heads.

Alice suggests that $P(C|A) = P(C|B)$.

$A = \{HT, HH\}$ with $P(A) = \frac{2}{4}$. \\
$B = \{HT, TH, HH\}$ with $P(B) = \frac{3}{4}$. \\
$C = \{HH\}$ with $P(C) = \frac{1}{4}$.

We will calculate both:

\begin{gather*}
    P(C|A) = \frac{P(C \cap A)}{P(A)} = \frac{\frac{1}{4}}{\frac{2}{4}} = \frac{1}{2}.\\
    P(C|B) = \frac{P(C \cap B)}{P(B)} = \frac{\frac{1}{4}}{\frac{3}{4}} = \frac{1}{3}.
\end{gather*}

We can generalize in the case that we have three events A, B, C with $A \subset B \subset C$, then the probability of A happening if we know that B happened is at least as probable as if we know that C happened.

\subsubsection*{Problem 16}

A: probability the coin toss result is heads: $P(A) = \frac{3}{6} = \frac{1}{2}$.

B: probability to pick the normal coin, with $P(B) = \frac{1}{3}$.

We want the probability of B given that A happened. 
So we have:

\begin{gather*}
    P(B|A) = \frac{P(B \cap A)}{P(A)} = \\
    \frac{\frac{1}{6}}{\frac{1}{2}} = \frac{1}{3}
\end{gather*}

Where $P(B \cap A)$ means that probability of getting a heads result and the coin is the normal one.

\subsubsection*{Problem 17}

We want the probability $P(A_1 \cap A_2 \cap A_3 \cap A_4)$, where $A_i$ means that the test is not defective.

So we can get this by calculating $P(A_1)P(A_2|A_1)P(A_3|A_1 \cap A_2)P(A_4| A_3 \cap A_2 \cap A_1)$.

\begin{gather*}
P(A_1) = \frac{95}{100} \\
P(A_2|A_1) = \frac{94}{99} \\
P(A_3|A_1 \cap A_2) = \frac{93}{98} \\
P(A_4| A_3 \cap A_2 \cap A_1) = \frac{92}{97}
\end{gather*}

So in total we have $\frac{95}{100} * \frac{94}{99} * \frac{93}{98} * \frac{92}{97} = 0.81$.

\subsubsection*{Problem 18}

\begin{gather*}
    P(A \cap B|B) = \\
    \frac{P(A \cap B \cap B)}{P(B)} = \\
    \frac{P(A \cap (B \cap B))}{P(B)} = \\
    \frac{A \cap B}{P(B)} = \\
    P(A|B)
\end{gather*}

\end{document}